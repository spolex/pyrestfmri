% !TEX encoding = UTF-8 Unicode
%!TEX root = memoria.tex
% !TEX spellcheck = es-ES
%%=========================================
\chapter{Desarrollo}
%%=========================================
\section{Estructura del experimento}
\subsection{Estructura de directorios}
\subsection{Configuración del experimento}
%%=========================================
\section{Módulo preprocesado}
\subsection{Parametrización}
\subsection{Salidas}
%%=========================================
\section{Módulo extracción de mapa cerebral}

CanICa eun un método ICA para el análisis de datos fMRI a nivel de grupo. Comparado con otras estrategias, aporta un modelo de grupo bien controlado, así como un algoritmo umbral que controla la especificidad y sensibilidad con un modelo explicito de la señal \cite{canica}


\subsection{Parametrización}
\subsection{Salidas}
%%=========================================
\section{Módulo extracción de regiones}
\subsection{Parametrización}
\subsection{Salidas}
%%=========================================
\section{Módulo para el cálculo de entropía}
\subsection{Parametrización}
\subsection{Salidas}
%%=========================================
\section{Persistencia e informe de los resultados}
\subsection{Parametrización}
\subsection{Salidas}
%%=========================================