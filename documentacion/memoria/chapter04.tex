% !TEX encoding = UTF-8 Unicode
%!TEX root = memoria.tex
% !TEX spellcheck = es-ES
%%=========================================
\chapter{Materiales}
%%=========================================

\section{Consideraciones éticas}

El uso de datos de caracter personal exige el cumplimiento del derecho a la protección de los datos personales en cumplimiento a la \hyperref[glos:lopd]{LOPD}. Es por esto que todos los datos han sido anonimizados de tal forma que se han alterado de los metadatos aquellos campos que puedan identificar univocamente a ningún individuo.

%%=========================================
\section{Pacientes para el experimento}

\subsection{Datos demográficos}

\subsection{Fuentes de origen}

\begin{enumerate}
\item Neuroimágen funcional: \textit{fmri}
\item Neuroimágen anatómica: \textit{MPRAGE}
\end{enumerate}

\section{Herramientas open source para el preprocesado de neuroimágen}

\subsection{Introducción a python}

\subsubsection{Numpy}
\subsubsection{Scipy}
\subsubsection{Matplotlib}

\subsection{Motor de flujos y preprocesado nipype}

\subsection{Procesado de neuroimagen FSL}

\subsection{Normalizado de imágen ANTs}

\subsection{Machine Learning para neuroimagen Nilearn}

\subsubsection{Introducción a sklearn}

\subsubsection{Extracción del mapa cerebral funcional}

\begin{itemize}
\item[FastICA]
\item[CanICA]
\item[DictLearning]
\end{itemize}

\subsection{Procesado de series temporales con nitime}