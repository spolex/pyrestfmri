% !TEX encoding = UTF-8 Unicode
%!TEX root = memoria.tex
% !TEX spellcheck = es-ES
%%=========================================
\chapter{Introducción}

La imagen funcional de resonancia magnética en \textit{Resting-state} ha sido ampliamente utilizada desde que en 1995 se presentó el primer informe basado en el análisis BOLD \ref{glos:bold} orientado a identificar actividad neuronal espontanea \cite{mattoolbox}. La imagen de resonancia magnética fMRI en estado de reposo se considerada una potente herramienta para investigar la actividad neuronal espontanea. También es recomendada para realizar estudios clínicos, ya que se obtiene una resolución espacial y temporal aceptable y la no invasividad, así como su simplicidad ya que no es necesario diseñar y realizar complicadas tareas cognitivas.

La conectividad funcional es uno de los análisis más utilizados en este tipo de estudios \cite{biswal,heuvel}. Todos estos métodos, y otros que aún no se han mencionado aquí, podrían formar parte de un framework de herramientas orientado al análisis de estudios fMRI en estado de reposo, ya que requieren de unas técnicas de preprocesado y análisis que puede no ser conveniente para otras modalidades. En este contexto y tras el estudio de las distintas herramientas se ha optado por usar el lenguaje de programación Python como tecnología principal, ya que actualmente existen proyectos importantes y suficientemente maduros, además de estar creados y mantenidos por una comunidad de neurocientificos que basa el desarrollo de estas herramientas en el estado del arte. Un claro ejemplo incluido en los \textbf{frameworks} utilizados es nipype \footnote{\url{nipype.readthedocs.io/en/latest/}} que dota la herramienta de un motor de workflows escalable, con conectores para las herramientas más importantes desarrollados y la posibilidad de paralelizar distintas etapas de los procesos entre otras bondades de las cuales veremos una pincelada en este trabajo. Otro pilar fundamental para el desarrollo de las mismás características, pero más orientado a tareas de aprendizaje automático es nilearn \footnote{http://nilearn.github.io/} basado en el popular framework de Machine Learning escrito en python Scikit-learn \footnote{url{http://scikit-learn.org/stable/}}.

TODO intro ESSENTIAL TREMOR¿? 
%%=========================================
\section{Motivación}

%%=========================================

\section{Objetivos de este Trabajo}

El análisis no invasivo del cerebro mediante neuroimagen, permite obtener valiosa información para la neurociencia y la medicina moderna. El presente estudio está orientado a la investigación y en la ayuda al diagnóstico precoz de la enfermedad de Temblor esencial. En este sentido se abordarán los siguientes objetivos:

\begin{itemize}
\item Investigar técnicas de procesado y análisis de imágen \hyperref[glos:fmri]{fMRi} en Temblor esencial y otros desordenes.
\item Investigación de las técnicas de construcción de redes cerebrales.
\item Análisis de la evolución de la intensidad en los mapas del fMRi
\item Investigar técnicas de análisis no lineal de neuroimagen,
\item Construción de una herramienta opensource modular, reutilizable y escalable orientada al análisis de neuroimagen
\end{itemize}

%%=========================================

\section{Estructura}
%%%=========================================
%\subsection*{Revisión bibliográfica}
%
%\subsubsection{Publicaciones en revistas internacionales indexadas en el \textit{Journal Citation Reports}}
%
%\begin{itemize}
%\item \fullcite{Yao2013}
%\item \fullcite{Li2016}
%\item \fullcite{Nicolini2016}
%\item \fullcite{Gastner2016}
%\item \fullcite{Guryanova2016}
%\item \fullcite{Grandy2016}
%\item \fullcite{Zamora-Lopez2016}
%\item \fullcite{Watanabe2013}
%\item \fullcite{Radulescu2013}
%\item \fullcite{Politis2014}
%\end{itemize}
