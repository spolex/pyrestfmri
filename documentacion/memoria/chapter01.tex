% !TEX encoding = UTF-8 Unicode
%!TEX root = memoria.tex
% !TEX spellcheck = es-ES
%%=========================================
\chapter{Introducción}

%%=========================================
\section{Motivación}

%%=========================================

\section{Objetivos de este Trabajo}

El análisis no invasivo del cerebro mediante neuroimagen, permite obtener valiosa información para la neurociencia y la medicina moderna. El presente estudio está orientado a la investigación y en la ayuda al diagnóstico precoz de la enfermedad de Temblor esencial. En este sentido se abordarán los siguientes objetivos:

\begin{itemize}
\item Investigar técnicas de procesado y análisis de imágen \hyperref[glos:fmri]{fMRi} en Temblor esencial y otros desordenes.
\item Investigación de las técnicas de construcción de redes cerebrales.
\item Análisis de la evolución de la intensidad en los mapas del fMRi
\item Investigar técnicas de análisis no lineal de neuroimagen,
\item Construción de una herramienta opensource modular, reutilizable y escalable orientada al análisis de neuroimagen
\end{itemize}

%%=========================================

\section{Estructura}
%%%=========================================
%\subsection*{Revisión bibliográfica}
%
%\subsubsection{Publicaciones en revistas internacionales indexadas en el \textit{Journal Citation Reports}}
%
%\begin{itemize}
%\item \fullcite{Yao2013}
%\item \fullcite{Li2016}
%\item \fullcite{Nicolini2016}
%\item \fullcite{Gastner2016}
%\item \fullcite{Guryanova2016}
%\item \fullcite{Grandy2016}
%\item \fullcite{Zamora-Lopez2016}
%\item \fullcite{Watanabe2013}
%\item \fullcite{Radulescu2013}
%\item \fullcite{Politis2014}
%\end{itemize}
