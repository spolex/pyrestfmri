% !TEX encoding = UTF-8 Unicode
%!TEX root = thesis.tex
% !TEX spellcheck = en-US
%%=========================================

\chapter{Glosario}
\begin{description}
\item[MRI] Imagen por resonancia magnética. \label{glos:mri}
\item[FMRI] Imagen por resonancia magnética funcional. \label{glos:fmri}
\item[SMRI] Imagen anatomica por resonancia magnetica. \label{glos:smri}
\item[NIFTI] Estandar de imágen médica. \label{glos:nifti}
\item[DICOM] Nuevo estandar de imágen médica. \label{glos:dicom}
\item[BOLD] Del inglés Blood oxygenation level-dependent. \label{glos:bold}
\item[TE] Del inglés Essential Tremor. \label{glos:et}
\item[SNR] Del inglés signal to noise ratio. Es el ratio o fuerza de la señal frente a la señal del ruido. \label{glos:snr}
\item[ICA] Del inglés Independent Component Analysis. Procesado de señal multivariable. \label{glos:ica}
\item[LOPD] Ley Orgánica 15/1999, de 13 de diciembre, de Protección de Datos de Carácter Personal
 \label{glos:ica}
\item[voxel] Píxel de un objeto3D
\item[bet] Brain Extraction Tool
\item[fsl]  FSL es una librería para el análisis de imágenes fMRI, MRI, DTI del cerebro \url{https://fsl.fmrib.ox.ac.uk/fsl/fslwiki/FSL}
\item[ANTs] Advanced Normalization Tools \url{http://stnava.github.io/ANTs/}
 \end{description}