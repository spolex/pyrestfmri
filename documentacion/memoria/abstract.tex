\thispagestyle{plain}
\begin{center}
    \Large
    \textbf{Tecnicas no invasivas de detección precoz de Temblor Esencial}
    
    \vspace{0.4cm}
    \large
    Preprocesado y análisis de neuroimagen fmri en estado de reposo
    
    \vspace{0.9cm}
    \textbf{Abstract}
\end{center}

En este Trabajo de Fin de Master, se ha estudiado la actividad BOLD espontánea en 49 pacientes con la enfermedad de Temblor Esencial (ET) y en 5 sujetos de control utilizando métodos de análisis no lineal. El objetivo principal de este estudio es determinar si la actividad cerebral BOLD es distinta en los enfermos y en los controles.
Dentro de las técnicas de neuroimágen nos encontramos encontramos con la imagen funcional por resonancia magnética \hyperref[glos:fmri]{fMRI} y la imágen anatómica \hyperref[glos:smri]{sMRI}, es una técnica no invasiva que permite el estudio basado en el fenomeno de resonancia magnética, que valora las características bioquímicas de los tejidos y que es especialmente efectivo a nivel cerebral. 
El entendimiento del funcionamiento del crebro humano es un desafio constante en el campo de la neurociencia. 
En áreas de psicología clínica, neurofisiología y neurociencias, es de interés describir cuantitativamente, así como cualitativamente, las funciones neuronales en condiciones consideradas normales y bajo la influencia diversos trastornos para posteriormente utilizar este conocimiento con fines de diagnóstico.
En el presente trabajo se comenzará por el estudio y aplicación de algunas de las técnicas de procesado de neuroimagen que actualmente están en desarrollo para finalmente realizar el estudio no lineal de las imágenes de afectados por la enfermedad de Temblor Esencial, con el fin de obtener más conocimiento sobre la influencia de esta enfermedad en la actividad cerebral en estado de reposo.  